\documentclass{article}
\usepackage[utf8x]{inputenc}
\usepackage[russian]{babel}
\usepackage{graphicx}
\usepackage{indentfirst}
\usepackage{amsmath}
\graphicspath{{pic/}}
\DeclareGraphicsExtensions{.pdf,.png,.jpg}
\author{Титовец}
\begin{document}
Титовец Юлия Алексеевна
\section{Параллельное программирование}
Параллельное программирование служит для создания программ, эффективно использующих вычислительные ресурсы за счет одновременного исполнения кода на нескольких вычислительных узлах.
\subsection{Модели программирования}
Совокупность приемов программирования, структур данных, отвечающих архитектуре гипотетического компьютера, предназначенного для выполнения определенного класса алгоритмов, называется моделью программирования.
\subsection{Последовательная модель программирования}
Традиционной считается последовательная модель программирования. В этом случае в любой момент времени выполняется только одна операция и только над одним элементом данных. Последовательная модель универсальна. Ее основными чертами являются применение стандартных языков программирования (для решения вычислительных задач это, обычно, Fortran и С/С++), хорошая переносимость программ и невысокая производительность. 
Основными особенностями модели параллельного программирования являются более высокая производительность программ, применение специальных приемов программирования и, как следствие, более высокая трудоемкость программирования, проблемы с переносимостью программ. Параллельная модель не обладает свойством универсальности. В параллельной модели программирования появляются проблемы, непривычные для программиста, привыкшего заниматься последовательным программированием. Среди них: управление работой множества процессоров, организация межпроцессорных пересылок данных и другие.
\newpage

Вставим какую-то формулу для умного вида вот прямо здесь:
\linebreak
\begin{equation}\label{eq1}
	\theta_0 =
	\begin{vmatrix}
		b_{11} &b_{12} &a_{13} &a_{14}\\
		b_{21} &b_{22} &a_{23} &a_{24}\\
		b_{31} &b_{32} &a_{33} &a_{34}\\
		b_{41} &b_{42} &a_{43} &a_{44}
	\end{vmatrix},\qquad
	A=
	\begin{pmatrix}
		a_{11} &a_{12} &b_{13} &b_{14}\\
		a_{21} &a_{22} &b_{23} &b_{24}\\
		a_{31} &a_{32} &b_{33} &b_{34}\\
		a_{41} &a_{42} &b_{43} &b_{44}
	\end{pmatrix}
\end{equation}

\begin{figure}
	\center{\includegraphics[scale=0.2]{1.jpg}}
	\caption{Какая-то картинка}
	\label{fig:image}
\end{figure}
\newpage
\tableofcontents



\end{document}